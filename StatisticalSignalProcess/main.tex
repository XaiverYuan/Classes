%search QUESTION
\documentclass{article}
\usepackage[utf8]{inputenc}
\usepackage{ctex}
\usepackage{amssymb}
\usepackage{bm}
\usepackage{ulem}
\usepackage{amsmath}
\title{统计信号处理HW4}
\author{袁宜桢}
\date{March 15, 2023}

\begin{document}
\maketitle

\section{4.1}
套用线性模型公式
$$ \bm{x}=\bm{H\theta}+\bm{w} $$
则$$ \hat{\theta}\sim N((\bm{H}^T\bm{H})^{-1}\bm{H}^T\bm{x},\sigma^2(\bm{H}^T\bm{H})^{-1}) $$
其中,在我们这道题中$$H=\left[\begin{array}{cc}
   1  & -1 \\
   1  & 1  \\
   \vdots  & \vdots \\
   1  & -1 \\
   1  & 1  \\
\end{array}\right]$$
$$ \bm{H}^T\bm{H}=\left[ \begin{array}{cc}
   N  & 0 \\
   0  & N
\end{array} \right] $$

$$ (\bm{H}^T\bm{H})^{-1}=\left[ \begin{array}{cc}
   \frac{1}{N}  & 0 \\
   0  & \frac{1}{N}
\end{array} \right] $$

$$ (\bm{H}^T\bm{H})^{-1}\bm{H}^T=\left[ \begin{array}{cccc}
   \frac{1}{N}  & \frac{1}{N} &\cdots &\frac{1}{N}  \\
   -\frac{1}{N}  & \frac{1}{N}&\cdots &\frac{1}{N}
\end{array} \right] $$
$$ \therefore \hat{\theta}=\left[ \begin{array}{cccc}
   \frac{1}{N}  & \frac{1}{N} &\cdots &\frac{1}{N}  \\
   -\frac{1}{N}  & \frac{1}{N}&\cdots &\frac{1}{N}
\end{array} \right]\bm{x}  $$
$$ Cov(\hat{\theta})=\left[ \begin{array}{cc}
   \frac{\sigma^2}{N}  & 0 \\
   0  & \frac{\sigma^2}{N}
\end{array} \right] $$
\section{4.14}
刚开始没看明白这个题目,去看了下答案,但是即使现在我还是觉得挺离谱的,就是正常不应该是衰落的概率是$\epsilon$,然后我们不知道它什么时候衰落,然后我们算期望啊什么的,虽然这样会更难,但是更intuitive?然后题目突然来一个要么不衰落,衰落就在M轮...?

看明白它的意思,我就可以自己写了
$$ \hat{A} = (\bm{H}^T\bm{H})^{-1}\bm{H}^T\bm{x} $$
\subsection{不衰落}
$$\bm{H}=\left[ 1,1,1,\cdots,1,1\right]^T$$
$$ \hat{A}=\frac{1}{N} \sum_{n=0}^{N-1}x[n]$$
\subsection{衰落} $$\bm{H}=\left[ 1,1,1\cdots,1,0,\cdots,0 \right]^T$$,其中M个1,N-M个0,
这里其实就相当于后面的数据被抹除了,就计算了前M个,intuitively,这也合理
$$ \hat{A}=\frac{1}{M} \sum_{n=0}^{M-1}x[n]$$
\subsection{方差}
$$ C_{\hat{\theta}}=\sigma^2E_H[(\bm{H}^T\bm{H})^{-1}]=\sigma^2(\epsilon*\frac{1}{M}+(1-\epsilon)*{\frac{1}{N}}) $$
而没有衰落的方差是$ \sigma^2*\frac{1}{N}=\sigma^2(\epsilon*\frac{1}{N}+(1-\epsilon)\frac{1}{N}) $
因为N大于M,所以有衰落的方差更大,这个其实很好理解,如果epsilon是1,它铁定衰落,那么个数少了,方差变大,而如果epsilon不是1,不衰减的时候一样,衰减的时候变烂,那么总的还是变烂
\section{6.4}
无论在那种情况下,$x$的均值都和$\mu$相同,所以$\bm{s}=[1,1,\cdots,1]^T$,而$ C $是对角矩阵,则
$$E[\hat{\theta}]= \frac{\bm{s^T}\bm{C}^{-1}E(x)}{\bm{s}^T\bm{C}^{-1}\bm{s}}=E[x] $$
在高斯分布的情况下,加的是高斯噪音,但是在拉普拉斯的情况下加的并不是高斯噪音(PPT4-25),所以高斯的BLUE是MVU,且是有效估计量,而另一个不是
\section{6.9}

在本题中,如果没有额外说明,则$\sum$均指代$\sum_{n=0}^{N-1}$
$$ \bm{H}=[\cos(0),\cos(2\pi f_1),\cos(4\pi f_1),\cdots,\cos(2\pi f_1n)]^T $$
$$ Cov(\hat{\theta})=\sigma^2\bm{I} $$
$$ \bm{H}^T\bm{C}^{-1}\bm{H}=\frac{1}{\sigma^2} \sum\cos^2(2\pi f_1 n)$$
$$ \bm{H}^T\bm{C}^{-1}\bm{x}=\frac{1}{\sigma^2} \sum\cos(2\pi f_1 n)x[n]$$
$$ \therefore \hat{\theta}=\frac{\sum\cos(2\pi f_1 n)x[n]}{\sum\cos^2(2\pi f_1 n)} $$
$$ \bm{C}_{\hat{\theta}}=(\bm{H}^T\bm{C}^{-1}\bm{H})^{-1}=\frac{\sigma^2}{\sum\cos^2(2\pi f_1 n) }$$
只有当$f_1$等于整数时,上述方差才会等于CRLB界,所以最好$f_1$取0.\\
我也觉得,那我也好算很多,搞半天不还是最简单的最有效
\section{6.14}
$$ p(w[n])=(1-\epsilon)N(0,\sigma_B^2)+\epsilon N(0,\sigma^2_I) $$
$$ \therefore Var(p(w[n]))=(1-\epsilon)\sigma_B^2+\epsilon\sigma_I^2 $$
\end{document}

